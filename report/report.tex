%% vi: set tabstop=2, set textwidth=80

\documentclass[11pt]{article}

\usepackage{homework}
\usepackage{graphicx}
\usepackage{algorithm}
\usepackage{algorithmic}
\usepackage{amsmath, amsthm, amssymb}
\usepackage{subfigure}
\usepackage[english]{babel}

% The report should be a description of your work during the lab sessions,
% focussing on the mean shift tracker. You should describe what you did, what you
% noticed and what you could have done different or could have improved. As you
% implemented the tracker you noticed different behaviour if you changed parts of
% the tracker, e.g. a different colour space or the number of bins in the
% histogram. Hopefully these insights have improved your tracker. This is why it
% is essential that you test the tracker also on a video of a domain other than
% soccer (or any other sport on a green field). This other domain will show how
% your implementation depends on the soccer domain. Observe how you can improve
% your design and them describe how you implemented this change or, for lack of
% time, describe how you would change your design. 

\title{Intelligent Multimedia Systems \\ Mean Shift Tracker}

\author{F. Huizinga [0418862], B. Stoeller [0426857] \\
      \{folkerthuizinga,bstoeller\}@gmail.com}

\date{December 27, 2009}

\begin{document}
\maketitle

\begin{abstract}
This report describes the implementation and results of a mean shift tracker
using the Epanechnikov kernel and various color space models. Furthermore we
perform some analysis on the tracker and color models with the use of two
videos within different domains.
\end{abstract}


\section{Introduction}
An object tracker consists of two major components, the \emph{Object Model} and
the \emph{Tracking Algorithm}. The object model is represented using a
histogram in some color space. The tracking algorithm is the Mean Shift
Algorithm. This report compares six different color spaces to represent the
Object Model and their performance is measured using two videos in different
domains (sport and nature). The report is organized as follows.  Section 2
describes the implementation of the algorithm. The experiments and results are
shown in Section 3. Section 4 presents our conclusions, and finally possible
improvements are shown in Section 5.


\section{Implementation}
For our implementation we used the Matlab programming language. We describe the
implementation of the tracking algorithm and the object model seperatly.

\subsection{Object Model}
To represent our object model we implemented the following color space models:
rgb, xyz, RGB, HSI and XYZ. We decided to use a fixed number of total bins
(i.e. the number of bins is not affected by the number of dimensions). More
precise, if $N$ is the total number of bins, then the number of bins $b$ for
each dimension $i$ is $b_i = N^{1/d}$, where $d$ is the total number of
dimensions.

\subsection{Tracking Algorithm}
Algorithm \ref{alg:mst} shows our implementation of the mean shift tracking
algorithm.
\begin{algorithm}
	\caption{MeanShiftTracker($V$, $n$)}
	\begin{algorithmic}[1]
	\REQUIRE The video $V$ with $n$ frames
	\STATE \dots
	\medskip
	\end{algorithmic}
\label{alg:mst}
\end{algorithm}


\section{Experiments and Results}
For our experiments we applied the tracker to two videos within different
domains, see Figure \ref{fig:videos}. Our total number of bins was set to $N
\approx 1000$, resulting in $b_i = 10$ for $d = 3$ and $b_i = 32$ for $d = 2$.
\begin{figure}[!ht]
\centering
%\includegraphics[width=9cm]{img/videos}
\caption{The left image shows the soccer domain where we track an orange
player, and the right image shows a scene from Earth, where we track a cheetah
chasing its prey.}
\label{fig:videos}
\end{figure}



\section{Conclusion}


\section{Future Work}
A lot can be improved in our current implementation. Some of these improvements are:
\begin{itemize}
\item{\emph{Scale}, compute the Epanechnikov kernel for various scales and select the best scale for each frame using the Bhattacharyya distance.}
\item{\emph{Model}, update the target model $q_u$ when the distance between candidate $p_u$ and $q_u$ is too large.}
\item{\emph{Kalman}, improve the tracker by applying a Kalman filter to get (even) better approximations in future frames.}
\item{\emph{Color}, on the manual selection of the object to track, one could perform some analysis to determine which color space is best suited for the target model. For example the color space model with the highest variance coul be used.}
\item{\emph{Spatial}, divide the kernel into several parts to store spatial information for more discriminative models.}
\end{itemize}
Unfortunatly, due to time constraints we were not able to apply these kind of improvements.

\end{document}
